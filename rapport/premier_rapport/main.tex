\documentclass[a4paper,12pt]{article}

\usepackage[french]{babel}
\usepackage[T1]{fontenc}
\usepackage[utf8]{inputenc}
\usepackage{listings}
\usepackage{xcolor}

\lstdefinestyle{customc}{
	breaklines=true,
	frame=none,
	language=C,
	basicstyle=\footnotesize\ttfamily,
	keywordstyle=\bfseries\color{green!40!black},
	commentstyle=\itshape\color{purple!40!black},
	identifierstyle=\color{blue},
	stringstyle=\color{orange},
	numberstyle=\tiny\color{gray!40!black},
	breakatwhitespace=true,
}

\lstdefinestyle{customasm}{
	frame=none,
	language=[x86masm]Assembler,
	basicstyle=\footnotesize\ttfamily,
	commentstyle=\itshape\color{purple!40!black},
	breakatwhitespace=true,
}

\lstset{style=customc}

\title{Rapport 6 semaines PFE}
\author{Matthieu Renard}

\begin{document}

\maketitle{} 

\vspace{15em}

\tableofcontents{}

\newpage{}

\section{Le LaBRI}

Le LaBRI (Laboratoire Bordelais de Recherche Informatique) est un laboratoire
dépendant de l'université de Bordeaux, situé sur le campus de l'ancienne
université de Bordeaux 1. Le laboratoire est divisé en six équipes, chacune
travaillant sur des aspects différents de l'informatique :
\begin{itemize}
\item méthodes formelles ;
\item combinatoire et algorithmique ;
\item image et son ;
\item modèles et algorithmes pour la bioinformatique et la visualisation
      d'informations ;
\item programmation, réseaux et systèmes ;
\item supports et algorithmes pour les applications.
\end{itemize}
Le LaBRI regroupe des chercheurs, des doctorants ainsi que des enseignants
chercheurs, qui enseignent à l'université ainsi que dans certaines écoles
de l'IPB (Institut Polytechnique de Bordeaux, qui devrait changer de nom
sous peu). L'équipe Méthodes Formelles à laquelle j'appartiens est
composée d'un peu plus d'une cinquantaine de personnes dont le but est
de faire évoluer l'informatique dans des sujets allant de la reconnaissance
du langage à la vérification de programmes en passant par l'étude des
graphes et de la logique. 


\section{Nature du travail}

Le travail à effectuer est avant tout un travail de recherche. Il s'agit
donc tout d'abord de s'informer sur l'état de l'art dans le domaine étudié,
afin de connaître l'avancée des recherches effectuées, puis de tenter 
d'en améliorer un ou plusieurs aspects. Si le temps le permet, il est 
également possible d'intégrer une part d'implémentation qui permettrait
entre autres d'évaluer le travail réalisé, en le comparant par exemple à
ce qui existait déjà.

\section{Environnement de travail}

Du fait de la nature du travail (essentiellement de la recherche), 
l'environnement de travail n'est pas prédéfini, il est donc possible
d'utiliser toutes les technologies souhaitées afin de mener à bien le
projet. Une salle au CREMI, un bâtiment de l'université, est dédiée aux
stagiaires de master recherche et dispose de machines en libre accès avec
une distribution Debian disposant d'un très grand nombre d'outils installés.
Il est également possible d'utiliser Windows sur des machines virtuelles 
installées sur ces ordinateurs, ou bien dans d'autres salles du CREMI.


\section{Sujet de stage}

Le stage porte sur la recherche de variables dans un programme binaire.
S'il est aisé de récupérer du code assembleur à partir d'un programme
binaire en connaissant l'architecture sur laquelle il s'exécute, la lecture
du code assembleur est fastidieuse, il est donc intéressant de l'étudier
de manière automatique, en tentant par exemple de le transformer en un
code de plus haut niveau équivalent, qui sera plus facilement compréhensible
par un être humain. \\
\indent{}Afin de mieux analyser le code assembleur, il est
intéressant de déterminer la position des variables, ainsi que leurs types.
Le type d'une variable peut être par exemple un type primitif : un entier
encodé sur un certain nombre de bits, ou un caractère (qui peut être
représenté par un entier), ou encore un type plus complexe comme une
structure (ensemble de variables), ou une union (variable dont le type peut
varier en fonction du temps). \\
\indent{}La rétroingénierie des programmes binaires a
plusieurs applications : tout d'abord cela permet de pouvoir adapter des
programmes propriétaires pour des systèmes "libres", mais aussi de
s'assurer que ces programmes ne contiennent aucun code malicieux ; cela permet
aussi de comprendre le fonctionnement d'un code malicieux afin de pouvoir
réparer les machines infectées ou le détecter avant qu'il n'infecte une
machine. Si certains programmes binaires contiennent des informations en
plus du code assembleur, comme le nom des variables ou leurs types, il
est impossible de garantir la validité de ces informations (s'il s'agit
de code malicieux, ces informations sont probablement manquantes ou 
corrompues), et une recherche de ces informations sur leur seule utilisation
a donc un réel intérêt.


\section{Les étapes nécessaires pour réaliser ce travail}

Afin de réaliser ce travail, il est nécessaire de se documenter sur ce qui
existe déjà dans le domaine de la recherche de types et plus généralement
dans la rétroingénierie de programmes binaires. Une fois cela réalisé,
il faut chercher un moyen d'améliorer ce qui existe, par exemple en
utilisant des structures de données plus adaptées, ou encore par
l'utilisation d'algorithmes et heuristiques ingénieux. Si le temps le permet,
une implémentation est envisageable, dans le programme \textit{insight} par 
exemple, développé par l'équipe de Méthodes Formelles du LaBRI, qui est
un logiciel distribué sous license BSD qui permet l'analyse statique de
programmes binaires. Ce logiciel est actuellement en cours de développement
et permet, entre autres, de récupérer le graphe de flot de contrôle
(Control Flow Graph, CFG) d'un programme. Il utilise une abstraction du code
assembleur, appelée "microcode" et qui est beaucoup plus verbeuse que
l'assembleur, puisqu'elle décompose tous les effets de bord d'une instruction
(mise à jour des drapeaux (flags) du processeur, etc...). L'analyse
devrait donc s'effectuer sur ce microcode, ce qui peut rendre la tâche
plus difficile ou la simplifier. 


\section{Avancement actuel}

Après avoir lu des articles abordant le problème sur différents niveaux,
de l'analyse s'intéressant seulement au typage de variables déjà connues
grâce à des équations sur les variables en forme SSA (Single Static 
Assigment), qui consiste à dupliquer les variables chaque fois qu'elles
sont modifiées, afin de garder une trace de chacune des valeurs qu'elle
prend, à la reconnaissance des variables dans le code assembleur, avec
distinction des structures et des tableaux, la "lecture" de code assembleur
a permis de se familiariser davantage avec ce langage, ses conventions,
et d'essayer de repérer des schémas, qui indiqueraient un même type 
de données par exemple. Cependant, ces schémas ne sont valables que lorsque
le code est généré par un même compilateur ; le but ici étant de tenter
d'analyser un code assembleur potentiellement malicieux et donc modifié
par un humain, il est impossible de présumer que de tels schémas sont
valables pour en déduire des informations certaines. Extraire des
informations certaines de code assembleur ne semble pas chose aisée, et il
serait intéressant de tenter d'en lister un maximum, afin de les utiliser
et de les compléter avec l'aide d'heuristiques. L'étude de code assembleur
montre rapidement qu'il est difficile d'avoir une certitude quant aux types
des variables en se basant uniquement sur leur utilisation. Il est possible
de distinguer des types primitifs d'autres en fonction des instructions
utilisées : les variables à virgule flottante disposent d'instructions
particulières, et se différencient donc immédiatement des variables
entières ; elles ne présentent donc que peu d'intérêt à l'étude, les
distinguer étant trivial. Il est souvent possible de déterminer la taille
d'une variable entière selon les opérandes utilisés (par exemple, sur une
architecture x86, les registres ont des noms différents en fonction de leur
taille : eax est sur 32 bits, ax sur 16 et ah sur 8). Il n'existe cependant
pas de distinction au niveau du processeur entre un pointeur et un entier,
la plupart des opérations arithmétiques pouvant s'effectuer sur l'un comme
sur l'autre. À ce propos, la norme du langage C définit précisément les
différentes opérations autorisées sur les pointeurs. Il est ainsi possible
d'additionner un pointeur avec un entier, de soustraire un entier à un
pointeur, et de soustraire un pointeur à un autre pointeur en C. Le but ici
est d'analyser du code assembleur qui a été
potentiellement modifié, voire intégralement écrit par un homme,
et non produit par un compilateur. Il peut arriver, par exemple, avoir besoin
d'assurer l'alignement d'un pointeur sur un certain nombre d'octets, ce qui
peut être implémenté à l'aide d'un "et" logique. Si cela est illégal en C,
il est néanmoins possible de l'effectuer, à l'aide d'un changement de type
temporaire. C'est pourquoi, afin de distinguer entiers et pointeurs, nous
nous sommes limités au seul déréférencement. Un second problème se pose :
même en assembleur, certains déréférencements se font par des calculs,
la plupart du temps entre un registre et une constante. Par exemple :
\begin{lstlisting}[style=customasm]
mov $0x0,-0x4(%ebp)
mov $0x0,0x80496a0(,%eax,4)
\end{lstlisting}
Dans le code ci-dessus, on peut voir deux cas différents de 
déréférencements : le premier est utilisé pour accéder à une variable
locale, alors que le second est un accès à un élément d'un tableau 
déclaré en tant que variable globale. Si la syntaxe utilisée ici
permet de distinguer les deux utilisations, ce n'est pas toujours le cas,
et sur certaines architectures, la seconde utilisation pourra avoir une
syntaxe équivalente à la première :
\begin{lstlisting}[style=customasm]
mov 0x0,0x2400(eax)
\end{lstlisting}
On accède ici à un élément d'un tableau qui est une variable globale, la
syntaxe étant proche de celle du processeur TI MSP430. Rien ne permet plus
de distinguer lequel du registre ou de la constante est un pointeur. Le
typage de ce type d'instructions ne peut donc résulter qu'en une formule
logique indiquant que l'un est un pointeur et l'autre un entier. Dans le
premier cas, c'est le registre \textit{ebp} qui est le pointeur, alors que
dans les autres cas, c'est la constante qui est un pointeur, le registre
\textit{eax} servant à parcourir le tableau.
\paragraph{}
La recherche de types complexes demande une analyse différente, qui ne repose
plus uniquement sur l'analyse des instructions. Pour repérer la présence
d'une structure, le moyen le plus évident est de constater un accès mémoire
sur une zone normalement inaccessible. En effet, lorsqu'on accède à un
champ d'une structure passée en paramètre à une fonction, si ce champ n'est
pas le premier de la structure, on accèdera à une zone mémoire qui est,
par exemple, quatre octets après le premier champ, soit quatre octets après
le pointeur passé en paramètre. L'accès à cette zone mémoire n'est 
possible que grâce à une contigüité en mémoire, qui est garantie par
l'utilisation d'une structure ou d'un tableau. Par ailleurs, la
représentation des types complexes est plus une question de choix que
d'une détermination objective. En effet, il est facile de produire deux
codes assembleur similaires mais utilisant des types différents. Il n'existe
aucune différence en assembleur entre une structure contenant trois entiers
et une autre contenant une structure de deux entiers et un entier. 
Cependant, le but est de rendre un code fastidieux à lire plus simple à
comprendre, c'est pourquoi chaque choix de représentation peut avoir des
avantages et des désavantages, et des choix différents peuvent finalement
 apporter des
informations de même qualité. Ce choix de typage peut même être plus ou
moins pertinent en fonction du lecteur, qui peut avoir une préférence pour
l'une ou l'autre des représentations. Tout ceci rend difficile l'évaluation
de la performance d'un algorithme de récupération de types, puisqu'il est
difficile d'être objectif sur la pertinence du choix des types de données
complexes.
\paragraph{}
L'objectif étant de typer des variables pour rendre un code assembleur plus
compréhensible, l'utilisation de concepts orientés objets est à envisager.
En effet, la programmation orientée objet est souvent vue comme étant de
haut niveau, ce qui signifie qu'elle est assez simple à comprendre pour un
relecteur. Les concepts qui peuvent être intéressants parmi les concepts
orientés objet sont l'héritage et le polymorphisme.
L'héritage permet de spécialiser un objet, ce qui signifie qu'on lui ajoute
des attributs tout en conservant ceux dont il dispose déjà. On peut ainsi
représenter grâce à un héritage une structure qui est toujours à
l'intérieur d'une autre structure. La structure contenue dans l'autre
correspond alors au parent de la structure qui la contient, qui est donc une
structure fille. Par exemple, dans le code ci-dessous, la structure
\textit{struct Parent} est considérée comme la structure parent de la
structure \textit{struct Fille} :
\begin{lstlisting}
struct Fille
{
	struct Parent
	{
		int a;
		int b;
	} parent;
	int c;
};
\end{lstlisting}
Cette notion d'héritage permet de faire apparaître la notion de
polymorphisme. Le polymorphisme est la capacité d'un objet à prendre
plusieurs formes, autrement dit, ici, la capacité d'une adresse à
représenter des objets de types différents. Dans le code ci-dessus, en 
effet, une variable de type \textit{struct Fille} aura la même adresse
que la structure \textit{struct Parent} qu'elle contient. Cela entraîne 
une difficulté pour différencier les deux types, comme le montre le code
 suivant :
\begin{lstlisting}
void init_Parent(struct Parent *p)
{
	p->a = 0;
	p->b = 0;
}

void init_Fille(struct Fille *f)
{
	init_Parent(&(f->parent));
	f->c = 0;
}
\end{lstlisting}
Les langages orientés objet permettent d'utiliser un objet comme s'il 
s'agissait d'un objet d'un type parent. On pourrait ainsi écrire, avec le
code de l'exemple précédent :
\begin{lstlisting}
void init_Fille(struct Fille *f)
{
	init_Parent(f);
	f->c = 0;
}
\end{lstlisting}
Le code est similaire, mis à part l'appel à \textit{init\_Parent} qui peut
se faire directement avec la variable f, puisqu'elle est considérée comme
étant d'un type qui hérite du type attendu par cette fonction. Cette syntaxe
est acceptable si l'héritage entre \textit{struct Parent} et \textit{struct
Fille} est bien mis en avant.
\paragraph{}
Une autre difficulté vient de la présence de conditions dans le code :
quelle valeur donner à une variable dont le résultat dépend d'une 
condition ? Le type même d'une telle variable pourrait dépendre de la
condition. Afin de s'abstraire de cette contrainte, nous avons décidé de
considérer séparément les différentes exécutions d'un programme. Cela
permet de connaître l'état du programme à tout moment, c'est-à-dire les
valeurs des registres et de la mémoire lors de l'exécution de n'importe 
quelle instruction. Le problème des conditions ne se pose donc plus, puisque
leur valeur est connue, il n'existe plus d'ambigüité sur les valeurs des
variables ni sur les branches empruntées par le programme.
\paragraph{}
Comme expliqué précédemment, la manière la plus simple de détecter une
structure est de repérer un accès à une zone mémoire normalement
inaccessible. Cette notion d'accessibilité se base sur celle de portée
d'une variable : on tente d'accéder à une zone mémoire qui ne devrait pas
être visible depuis la fonction actuelle. L'inconvénient est qu'on suppose
ici que les limites des fonctions sont connues, ce qui pourrait ne pas être
le cas. Une des questions qui se pose est alors de savoir s'il est possible
d'obtenir les informations du graphe d'appels, qui relie les lieux d'où
les fonctions sont appelées aux fonctions, et permet donc dans notre
cas de typer les paramètres et les valeurs de retour des fonctions. Notre
approche, qui considère chaque exécution du programme séparément semble
permettre de retrouver ces informations, car les paramètres sont passés
soit par registres, soit sur la pile, et dans les deux cas, les valeurs sont
connues.


\section{Conclusion}

Si les premières semaines passées à lire des articles pour tenter
d'en extraire des informations intéressantes furent quelque peu 
laborieuses, elles ont néanmoins permis d'aborder le sujet sous différents
angles qui me seront utiles pour la suite. Si l'implémentation n'est pas
une obligation, elle me semble intéressante car elle permettrait de mettre
en oeuvre les résultats que je pourrais trouver et ainsi valider ou infirmer
ceux-ci. 

\end{document}



