\documentclass[a4paper,12pt]{article}

\usepackage[french]{babel}
\usepackage[T1]{fontenc}
\usepackage[utf8]{inputenc}

\title{Rapport 6 semaines PFE}
\author{Matthieu Renard}

\begin{document}

\maketitle{} 

\vspace{15em}

\tableofcontents{}

\newpage{}

\section{Le LaBRI}

Le LaBRI (Laboratoire Bordelais de Recherche Informatique) est un laboratoire
dépendant de l'université de Bordeaux, situé sur le campus de l'ancienne
université de Bordeaux 1. Le laboratoire est divisé en six équipes, chacune
travaillant sur des aspects différents de l'informatique :
\begin{itemize}
\item méthodes formelles ;
\item combinatoire et algorithmique ;
\item image et son ;
\item modèles et algorithmes pour la bioinformatique et la visualisation
      d'informations ;
\item programmation, réseaux et systèmes ;
\item supports et algorithmes pour les applications.
\end{itemize}
Le LaBRI regroupe des chercheurs, des doctorants ainsi que des enseignants
chercheurs, qui enseignent à l'université ainsi que dans certaines écoles
de l'IPB (Institut Polytechnique de Bordeaux, qui devrait changer de nom
sous peu). L'équipe Méthodes Formelles à laquelle j'appartiens est
composée d'un peu plus d'une cinquantaine de personnes dont le but est
de faire évoluer l'informatique dans des sujets allant de la reconnaissance
du langage à la vérification de programmes en passant par l'étude des
graphes et de la logique. 


\section{Nature du travail}

Le travail à effectuer est avant tout un travail de recherche. Il s'agit
donc tout d'abord de s'informer sur l'état de l'art dans le domaine étudié,
afin de connaître l'avancée des recherches effectuées, puis de tenter 
d'en améliorer un ou plusieurs aspects. Si le temps le permet, il est 
également possible d'intégrer une part d'implémentation qui permettrait
entre autres d'évaluer le travail réalisé, en le comparant par exemple à
ce qui existait déjà.

\section{Environnement de travail}

Du fait de la nature du travail (essentiellement de la recherche), 
l'environnement de travail n'est pas prédéfini, il est donc possible
d'utiliser toutes les technologies souhaitées afin de mener à bien le
projet. Une salle au CREMI, un bâtiment de l'université, est dédiée aux
stagiaires de master recherche et dispose de machines en libre accès avec
une distribution Debian disposant d'un très grand nombre d'outils installés.
Il est également possible d'utiliser Windows sur des machines virtuelles 
installées sur ces ordinateurs, ou bien dans d'autres salles du CREMI.


\section{Sujet de stage}

Le stage porte sur la recherche de variables dans un programme binaire.
S'il est aisé de récupérer du code assembleur à partir d'un programme
binaire en connaissant l'architecture sur laquelle il s'exécute, la lecture
du code assembleur est fastidieuse, il est donc intéressant de l'étudier
de manière automatique, en tentant par exemple de le transformer en un
code de plus haut niveau équivalent, qui sera plus facilement compréhensible
par un être humain. \\
\indent{}Afin de mieux analyser le code assembleur, il est
intéressant de déterminer la position des variables, ainsi que leurs types.
Le type d'une variable peut être par exemple un type primitif : un entier
encodé sur un certain nombre de bits, ou un caractère (qui peut être
représenté par un entier), ou encore un type plus complexe comme une
structure (ensemble de variables), ou une union (variable dont le type peut
varier en fonction du temps). \\
\indent{}La rétroingénierie des programmes binaires a
plusieurs applications : tout d'abord cela permet de pouvoir adapter des
programmes propriétaires pour des systèmes "libres", mais aussi de
s'assurer que ces programmes ne contiennent aucun code malicieux ; cela permet
aussi de comprendre le fonctionnement d'un code malicieux afin de pouvoir
réparer les machines infectées ou le détecter avant qu'il n'infecte une
machine. Si certains programmes binaires contiennent des informations en
plus du code assembleur, comme le nom des variables ou leurs types, il
est impossible de garantir la validité de ces informations (s'il s'agit
de code malicieux, ces informations sont probablement manquantes ou 
corrompues), et une recherche de ces informations sur leur seule utilisation
a donc un réel intérêt.


\section{Les étapes nécessaires pour réaliser ce travail}

Afin de réaliser ce travail, il est nécessaire de se documenter sur ce qui
existe déjà dans le domaine de la recherche de types et plus généralement
dans la rétroingénierie de programmes binaires. Une fois cela réalisé,
il faut chercher un moyen d'améliorer ce qui existe, par exemple en
utilisant des structures de données plus adaptées, ou encore par
l'utilisation d'algorithmes et heuristiques ingénieux. Si le temps le permet,
une implémentation est envisageable, dans le programme \textit{insight} par 
exemple, développé par l'équipe de Méthodes Formelles du LaBRI, qui est
un logiciel distribué sous license BSD qui permet l'analyse statique de
programmes binaires. Ce logiciel est actuellement en cours de développement
et permet, entre autres, de récupérer le graphe de flot de contrôle
(Control Flow Graph, CFG) d'un programme. Il utilise une abstraction du code
assembleur, appelée "microcode" et qui est beaucoup plus verbeuse que
l'assembleur, puisqu'elle décompose tous les effets de bord d'une instruction
(mise à jour des drapeaux (flags) du processeur, etc...). L'analyse
devrait donc s'effectuer sur ce microcode, ce qui peut rendre la tâche
plus difficile ou la simplifier. 


\section{Avancement actuel}

Après avoir lu des articles abordant le problème sur différents niveaux,
de l'analyse s'intéressant seulement au typage de variables déjà connues
grâce à des équations sur les variables en forme SSA (Single Static 
Assigment), qui consiste à dupliquer les variables chaque fois qu'elles
sont modifiées, afin de garder une trace de chacune des valeurs qu'elle
prend, à la reconnaissance des variables dans le code assembleur, avec
distinction des structures et des tableaux, la "lecture" de code assembleur
a permis de se familiariser davantage avec ce langage, ses conventions,
et d'essayer de repérer des schémas, qui indiqueraient un même type 
de données par exemple. Cependant, ces schémas ne sont valables que lorsque
le code est généré par un même compilateur ; le but ici étant de tenter
d'analyser un code assembleur potentiellement malicieux et donc modifié
par un humain, il est impossible de présumer que de tels schémas sont
valables pour en déduire des informations certaines. Extraire des
informations certaines de code assembleur ne semble pas chose aisée, et il
serait intéressant de tenter d'en lister un maximum, afin de les utiliser
et de les compléter avec l'aide d'heuristiques. Une évaluation de la
difficulté apportée par le microcode par rapport à l'assembleur et 
l'intégration d'une analyse de type dans le logiciel \textit{insight} est
également à prévoir.

\section{Conclusion}

Si les premières semaines passées à lire des articles pour tenter
d'en extraire des informations intéressantes furent quelque peu 
laborieuses, elles ont néanmoins permis d'aborder le sujet sous différents
angles qui me seront utiles pour la suite. Si l'implémentation n'est pas
une obligation, elle me semble intéressante car elle permettrait de mettre
en oeuvre les résultats que je pourrais trouver et ainsi valider ou infirmer
ceux-ci. 

\end{document}



