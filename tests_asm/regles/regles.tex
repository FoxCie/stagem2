\documentclass[a4paper,11pt]{article}

\usepackage[french]{babel}
\usepackage[T1]{fontenc}
\usepackage[utf8]{inputenc}
\usepackage[french]{algorithm2e}
\usepackage{listings}
\usepackage{xcolor}
\usepackage{multicol}
\usepackage[margin=2cm]{geometry}


\lstdefinestyle{customc}{
	breaklines=true,
	frame=none,
	language=C,
	basicstyle=\footnotesize\ttfamily,
	keywordstyle=\bfseries\color{green!40!black},
	commentstyle=\itshape\color{purple!40!black},
	identifierstyle=\color{blue},
	stringstyle=\color{orange},
	numberstyle=\tiny\color{gray!40!black},
	breakatwhitespace=true,
}

\lstdefinestyle{customasm}{
	frame=L,
	language=[x86masm]Assembler,
	basicstyle=\footnotesize\ttfamily,
	commentstyle=\itshape\color{purple!40!black},
	breakatwhitespace=true,
}

\lstset{style=customc}

\title{Détection des types dans un programme}
\author{Matthieu Renard}

\begin{document}
\maketitle{}

\newpage

\section{Exemples de programmes}
\subsection{Détection de pointeurs}

On détermine qu'une variable est un pointeur s'il y a déréférencement :

\begin{multicols}{2}
\lstinputlisting[style=customc]{pointeurs/regle3.c}

\columnbreak

\lstinputlisting[style=customasm]{pointeurs/regle3_dump}
\end{multicols}

\newpage

\subsection{Détection de tableaux}

\begin{multicols}{2}
\lstinputlisting[style=customc]{tableaux/regle2.c}

\columnbreak

\lstinputlisting[style=customasm]{tableaux/regle2_dump}
\end{multicols}


\newpage

\subsection{Détection de structures}
\subsubsection{En paramètre à une fonction}
\begin{multicols}{2}
\lstinputlisting[style=customc]{struct/regle1.c}

\columnbreak

\lstinputlisting[style=customasm]{struct/regle1_dump}
\end{multicols}

\newpage

\subsubsection{Dans un tableau}
\begin{multicols}{2}
\lstinputlisting[style=customc]{struct/regle4.c}

\columnbreak

\lstinputlisting[style=customasm]{struct/regle4_dump}
\end{multicols}

\newpage

\subsubsection{Détection d'une liste chainée}
\begin{multicols}{2}
\lstinputlisting[style=customc]{struct/regle5.c}

\columnbreak

\lstinputlisting[style=customasm]{struct/regle5_dump}
\end{multicols}

\newpage

\subsubsection{Le problème des codes équivalents}
\begin{multicols}{2}
\lstinputlisting[style=customc]{struct/regle6.c}

\columnbreak

\lstinputlisting[style=customasm]{struct/regle6_dump}
\end{multicols}

\newpage

\subsubsection{Différenciation de structures}
\begin{multicols}{2}
\lstinputlisting[style=customc]{struct/regle7.c}

\columnbreak

\lstinputlisting[style=customasm]{struct/regle7_dump}
\end{multicols}

\newpage

\section{Algorithme}
\begin{algorithm}[H]
\SetKwInOut{Input}{input}\SetKwInOut{Output}{output}
\SetKwFor{For}{pour}{faire}{fin}

\Input{Un programme binaire p}
\Output{L'ensemble des types des variables du programme}
\BlankLine

$types \leftarrow $ tableau dont les indices sont les adresses du programme\;
\For{Chaque execution $s$ de $p$}{
	$types \leftarrow $ typer\_execution($s$, $types$)\;
}

\end{algorithm}

\end{document}

